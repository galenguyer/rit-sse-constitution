% BEFORE CHANGES ARE MADE TO THIS DOCUMENT:
% -References will be automatically updated if any part is added, deleted, etc.
%  However, if a sub part is moved to a different part, its references must be
%  changed.

\documentclass[american]{article}

\usepackage{hyperref}
\usepackage{titlesec}
\usepackage[iso]{isodate}
\usepackage{indentfirst}

% Fix margins
\setlength{\evensidemargin}{0in}
\setlength{\oddsidemargin}{0in}
\setlength{\textwidth}{6.5in}
\setlength{\topmargin}{0in}
\setlength{\textheight}{8.5in}

\setcounter{tocdepth}{4}
\setcounter{secnumdepth}{4}

% Title page information
\title{Society of Software Engineers Constitution}
\author{Society of Software Engineers}
% Last Modified Date
\newcommand{\datechanged}{Last Updated: \today}
\date{\datechanged}


\begin{document}
% Title
\maketitle

\section{Mission Statement}
The Society of Software Engineers (SSE) is a student organization dedicated to fostering academic and professional success. Its focus is on four fundamental pillars: mentoring, networking, projects, and events. Through these aspects, the Society strives to build relationships with all students, the department, other organizations, and the institute.

\section{Non-discrimination Policy}
It is the shared belief amongst members of the SSE that discrimination on the basis of sex, race, age, gender, sexual orientation/identity, personal preferences, or any other factor is not acceptable. The members of the Society commit to upholding tolerance and respect of all people regardless of their lifestyle or background.

\section{Code of Conduct}
As active members of the RIT community, advocates of software engineering, and future workers in the computing industry, the Society's members are expected to demonstrate the highest levels of personal and professional conduct.

At the core of all efforts as Society members, we will:

\begin{itemize}
\item Demonstrate civility, respect, decency and sensitivity towards our fellow members of the RIT \\ community.
\item Conduct ourselves with the highest standards of moral and ethical behavior. Such behavior includes taking responsibility for our own personal choices, decisions and academic and professional work.
\item Affirm through the daily demonstration of these ideals that RIT is a university devoted to the pursuit of knowledge and a free exchange of ideas in an open and respectful environment.
\end{itemize}

\section{Pillars of the Society}

The Pillars of the Society describe the organization's foremost priorities. The Pillars do not directly map to committees -- all actions taken by the Society should reflect these priorities and improve the Society within these areas.

The Pillars shall be valued equally, and are listed here in no particular order.

\subsection{Mentoring}
The Society shall promote the academic success of members of the RIT community, with a focus on material related to Software Engineering.

\subsection{Networking}
The Society shall maintain relationships with prospective employers, sponsors, students, alumni, and important members of the computing industry.

\subsection{Projects}
The Society shall provide students with the opportunity to improve their engineering and creative skills in a non-academic setting.

\subsection{Events}
The Society shall provide opportunities to promote friendship and fellowship within the organization.

\section{Membership}

\subsection{Active Membership}

\subsubsection{Rights}
Active members of the Society are permitted to vote in elections, start and sign petitions for the removal of officers, and are granted any other rights or responsibilities determined by the Primary Officers.

\subsubsection{Requirements}
Active membership shall only last for one academic term and must be re-earned each academic term the candidate wishes to be considered a member. Candidates who have been considered active members for a given academic term shall be considered active members for the first two weeks of the following term. This grace period shall allow for the acquisition of membership without the loss of any benefits.

A candidate must satisfy all of the following requirements to be an active member: 

\begin{enumerate}
\item The candidate must be enrolled as a student at RIT or NTID.
\item The candidate must make at least one significant contribution to the betterment or continuation of the Society. The details of a "significant contribution" shall be determined by the current Primary Officers' Policy.
\end{enumerate}

Students who are active members in the term prior to participating in co-op, study abroad, other student learning opportunities, medical leave, or other justified absence from campus as determined by a majority vote of the Primary Officers, shall maintain their membership status for the duration of their time away.

\subsection{Alumni Membership}

\subsubsection{Rights}
Alumni members of the SSE are not permitted to participate in votes, elections, or petitions. All other rights shall be determined by the Primary Officers.

\subsubsection{Requirements}
Alumni membership shall last indefinitely after acquisition unless directly contested. A candidate must satisfy all of the following requirements to have Alumni membership. 

\begin{enumerate}
\item The candidate must not be enrolled as a student at RIT or NTID.
\item The candidate must have earned Active membership while at RIT or NTID.
\item The candidate must have applied for alumni membership within six months of leaving RIT or NTID. Alumni membership can also be granted after the six month time period upon approval of the Primary Officers. The Officers' Policy shall specify the application process for Alumni membership.
\end{enumerate}

\subsection{Temporary Membership}

\subsubsection{Rights}
Temporary members of the SSE are not permitted to participate in votes, elections, or petitions. These members shall not be included in any count of the Society's members for the purposes of voting, elections, or petitions. All other rights shall be determined by the Primary Officers.

\subsubsection{Requirements}
Temporary membership may be granted for at most one academic term at the discretion of the Primary Officers if the following additional requirements are met:

\begin{enumerate}
\item The candidate must be enrolled as a student at RIT or NTID.
\item The candidate must not ever have had a Temporary, Active, or Alumni membership in the Society.
\end{enumerate}

\subsection{Removal of Membership}
All forms of membership may be contested at any time by the Primary Officers.

Membership may be removed only by a vote among the Primary Officers. Should a Primary Officer feel a member is not acting in the spirit of the SSE Constitution and its ideals, it is the responsibility of the Primary Officer to start a discussion with the other Primary Officers detailing the nature of the member's actions and reasons for removal of membership. This discussion shall be concluded with a vote among the Primary Officers for or against the placement of the candidate on a probationary period, the length of which shall be determined at the discretion of the Primary Officers based on the nature of the candidate's behavior. Should a decision in favor of probation be reached, the candidate shall be informed of the verdict and their probationary period shall begin. Should this not be reached, the candidate shall informed of the nature of the discussion.

At the end of the probationary period, a second discussion shall be held between the Primary Officers and the candidate detailing the improvement of the candidate during their probationary period. This discussion shall be concluded with a vote among the Primary Officers for or against the removal of membership from the candidate. If the officers vote in favor of removal, the candidate shall be informed of the verdict and be removed from the membership roster immediately. The former member shall also be barred from obtaining membership for a minimum of the remainder of the current academic term.

\section{Governance}
The collection of Primary Officers and Committee Heads, referred to as officers, is responsible for the initiation and execution of ideas and discussions that further the goals of the Society.

\subsection{Primary Officers}
The Primary Officers are leaders elected by the Society's members to direct the organization. Detailed responsibilities for each position shall be recorded in the Primary Officers' Policy.

\subsubsection{Positions}
\paragraph{President}
The President is responsible for directing the progress of the Society. As the leader of the organization, they are responsible to see that the society improves and progresses over the course of their presidency.
\paragraph{Vice President}
The Vice President's responsibilities shall support the president and their goals as they pertain to the Society.
\paragraph{Secretary}
The role of the secretary shall record the events and occurrences of the Society. They are responsible for the documentation of all useful knowledge accumulated throughout their term. In addition, the Secretary is responsible to see that necessary parties have easy and constant access to relevant information.
\paragraph{Treasurer}
The treasurer is responsible for all financial that happen on behalf of the SSE. As such, they must grant members the permission to use SSE funds, ensure all financial transactions are recorded properly, and see that funds and items purchased are used in a responsible way. 

\subsubsection{Period of Office}
For annual elections, the period of office begins at the end of the spring academic term. For any other elections, the period of office begins immediately. Periods of office conclude at the end of the spring academic term in the year for which the officer serves. Officers who no longer meet the candidacy requirements for office must resign.

\subsubsection{Impeachment}
Primary officers may only be removed by the completion of a petition started by an active member of the Society. Should a petition receive two thirds of the signatures of all active members, the Primary Officer shall be removed from their position immediately. Following the removal of the Primary Officer, a vote shall be held to fill the vacant position. In the time between the removal of the officer and the appointment of the new officer, all responsibilities of the removed Primary Officer shall be completed by the remaining Primary Officers.

\subsubsection{Resignation}
Primary Officers may resign at any point during their period of office. Upon resignation, the officer is stripped of the powers, rights and responsibilities of the position. Before the next election occurs, all responsibilities of the vacant position shall be fulfilled by the remaining Primary Officers. 

\subsection{Committees}

Committees may be formed to delegate responsibility for tasks within the Society. The internal structure of committees shall be specified in the Primary Officer's Policy. Committee Heads shall be appointed by Primary Officers, and shall be listed in the Primary Officers' Policy each year. No committee may have more than one Head at any time.

\subsubsection{Committee Heads}
Committee Heads are leaders that handle tasks and responsibilities related to a specific committee of the Society. The responsibilities and goals of each Committee Head are also listed in the Primary Officers' Policy.

Committee Head appointments terminate with the period of office of the primary officers who chose them.

\subsubsection{Committee Head Impeachment}
Committee heads may be removed by the Primary Officers for failing to fulfill the responsibilities of their position. The Primary Officers may vote to place the Committee Head on a probationary period, the length of which should be determined based on the nature of the Committee Head's behavior. Should a decision be reached, the candidate shall be informed of the verdict and the probationary period shall begin.

At the end of the probationary period, a discussion shall be held among the Primary Officers addressing behavior during the probationary period. The Primary Officers must make a reasonable effort to include the Committee Head in this discussion. This discussion shall be concluded with a vote among the Primary Officers for or against the removal of the candidate from their position; should a decision be reached, the candidate shall be informed of the verdict and removed from the position.

The probationary period may be skipped if and only if the Primary Officers come to a unanimous decision to skip probation and proceed directly to the removal of the Committee Head. This final clause should only be used in truly extraneous circumstances with a reasonable effort to allow the Committee Head to be present during the discussion and vote.

\subsubsection{Resignation}
Committee Heads may resign at any point during their period of office. Upon resignation, the head is stripped of the powers, rights and responsibilities of the position.

\section{Primary Officer Elections}
Elections for Primary Officers shall be held each year at the end of the Spring term (annual elections) and at any time when an Officer position is left unfilled. Only one member can occupy any elected position at a time.

\subsection{Candidacy and Eligibility}
Candidates for Primary Officer roles in the Society must meet the following requirements by the start of the term in which they would take office:

\begin{table}[h]
\centering
\begin{tabular}{|l|l|l|}
\hline
\textbf{Position} & \textbf{Year Level} & \textbf{Prior Involvement} \\ \hline
President & 4 & 2 academic terms \\ \hline
Vice President & 3 & 1 academic term \\ \hline
Treasurer & - & 1 academic term \\ \hline
Secretary & - & - \\ \hline
\end{tabular}
\end{table}

“Year level” is counted by the lower of [A] years enrolled at least part-time in a higher-education institution post high school graduation, or [B] by RIT credits obtained following the scale published by the RIT Office of the Registrar. “Prior Involvement” refers to the number of complete academic terms during which the candidate served as a Primary Officer or Committee Head. In the event a candidate has not previously served as a Primary Officer or Committee Head, two complete academic terms during which the candidate served as a Mentor with the society may be substituted to serve as one term of “Prior Involvement”. No more than one term of “Prior Involvement” earned through mentorship may be substituted as the requirements for any office.

All Primary Officers must be enrolled as full or part-time Software Engineering students attending classes at the RIT campus in Rochester, NY.

Candidates for President and Vice President must both plan to remain enrolled and on-campus for the entire academic year for which they are to serve. Candidates for Secretary and Treasurer must both plan to remain enrolled and on-campus for at least the most immediate academic term for which they are to serve.

\subsection{Collection and Acceptance of Nominations}

\subsubsection{Position Nominations}
The nomination of elected position candidates must begin and end at least 48 hours before elections are held. Any currently active member of the society can nominate eligible students for any electable position. Alumni members cannot submit nominations for Primary Officers.

\subsubsection{Nomination Notification/Acceptance}
All members of the Society that are nominated for an elected position must be notified of the nomination within 24 hours of their nomination in a written or electronic form. Acceptance or rejection of the nomination must then be made, either in written or electronic form, up to 24 hours before elections are held. In the event that a nominated candidate does not respond to a nomination notification within the time period, the nominated candidate will no longer be considered eligible for the position for the ongoing election process.

An up-to-date list of all accepted nominations must be made public and easily accessible.

\subsubsection{Election Process}
The election process for Primary Officers utilizes an instant-runoff voting system, where each nominee is ranked from 1 (least desired) to N (most desired), where N is the total number of nominees running for each position. In the event that a nominee is not ranked on a ballot, they will be considered to be ranked in last place (1) for that voter.

At the end of the voting period, the first choice vote for all voters is tallied. The candidate with the lowest number of votes is eliminated from the running. The votes are then re-tallied. Voters who selected a candidate who has been removed from the running instead use their next most favorable candidate. This process repeats until only one candidate remains. This remaining candidate shall be considered the winner.

In the event that one or more nominees have an equal sum ranking, the nominee with the highest average ranking is considered the winner. In the event that both the sum and average rank is equal, another vote consisting of the tied nominees will take place. This process repeats until a tie no longer exists.

Any active member of the Society can vote for an elected officer. Alumni or temporary members cannot participate in this process.

\end{document}