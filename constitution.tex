% BEFORE CHANGES ARE MADE TO THIS DOCUMENT:
% -References will be automatically updated if any part is added, deleted, etc.
%  However, if a sub part is moved to a different part, its references must be
%  changed.

\documentclass[american]{article}

\usepackage{hyperref}
\usepackage{titlesec}
\usepackage[iso]{isodate}
\usepackage{indentfirst}

% Fix margins
\setlength{\evensidemargin}{0in}
\setlength{\oddsidemargin}{0in}
\setlength{\textwidth}{6.5in}
\setlength{\topmargin}{0in}
\setlength{\textheight}{8.5in}

% Title page information
\title{Society of Software Engineers Constitution}
\author{Society of Software Engineers}
% Last Modified Date
\newcommand{\datechanged}{Last Updated: \today}
\date{\datechanged}


\begin{document}
% Title
\maketitle

\section{Mission Statement}
The Society of Software Engineers (SSE) is a student organization dedicated to fostering academic and professional success. Its focus is on four fundamental pillars: mentoring, networking, projects, and events. Through these aspects, the Society strives to build relationships with all students, the department, other organizations, and the institute.

\section{Non-discrimination Policy}
It is the shared belief amongst members of the SSE that discrimination on the basis of sex, race, age, gender, sexual orientation/identity, personal preferences, or any other factor is not acceptable. The members of the Society commit to upholding tolerance and respect of all people regardless of their lifestyle or background.

\section{Code of Conduct}
As active members of the RIT community, advocates of software engineering, and future workers in the computing industry, the Society's members are expected to demonstrate the highest levels of personal and professional conduct.

At the core of all efforts as Society members, we will:

\begin{itemize}
\item Demonstrate civility, respect, decency and sensitivity towards our fellow members of the RIT \\ community.
\item Conduct ourselves with the highest standards of moral and ethical behavior. Such behavior includes taking responsibility for our own personal choices, decisions and academic and professional work.
\item Affirm through the daily demonstration of these ideals that RIT is a university devoted to the pursuit of knowledge and a free exchange of ideas in an open and respectful environment.
\end{itemize}

\section{Pillars of the Society}

The Pillars of the Society describe the organization's foremost priorities. The Pillars do not directly map to committees -- all actions taken by the Society should reflect these priorities and improve the Society within these areas.

The Pillars shall be valued equally, and are listed here in no particular order.

\subsection{Mentoring}
The Society shall promote the academic success of members of the RIT community, with a focus on material related to Software Engineering.

\subsection{Networking}
The Society shall maintain relationships with prospective employers, sponsors, students, alumni, and important members of the computing industry.

\subsection{Projects}
The Society shall provide students with the opportunity to improve their engineering and creative skills in a non-academic setting.

\subsection{Events}
The Society shall provide opportunities to promote friendship and fellowship within the organization.

\section{Membership}

\subsection{Active Membership}

\subsubsection{Rights}
Active members of the Society are permitted to vote in elections, start and sign petitions for the removal of officers, and are granted any other rights or responsibilities determined by the Primary Officers.

\subsubsection{Requirements}
Active membership shall only last for one academic term and must be re-earned each academic term the candidate wishes to be considered a member. Candidates who have been considered active members for a given academic term shall be considered active members for the first two weeks of the following term. This grace period shall allow for the acquisition of membership without the loss of any benefits.

A candidate must satisfy all of the following requirements to be an active member: 

\begin{enumerate}
\item The candidate must be enrolled as a student at RIT or NTID.
\item The candidate must make at least one significant contribution to the betterment or continuation of the Society. The details of a "significant contribution" shall be determined by the current Primary Officers' Policy.
\end{enumerate}

Students who are active members in the term prior to participating in co-op, study abroad, other student learning opportunities, medical leave, or other justified absence from campus as determined by a majority vote of the Primary Officers, shall maintain their membership status for the duration of their time away.

\subsection{Alumni Membership}

\subsubsection{Rights}
Alumni members of the SSE are not permitted to participate in votes, elections, or petitions. All other rights shall be determined by the Primary Officers.

\subsubsection{Requirements}
Alumni membership shall last indefinitely after acquisition unless directly contested. A candidate must satisfy all of the following requirements to have Alumni membership. 

\begin{enumerate}
\item The candidate must not be enrolled as a student at RIT or NTID.
\item The candidate must have earned Active membership while at RIT or NTID.
\item The candidate must have applied for alumni membership within six months of leaving RIT or NTID. Alumni membership can also be granted after the six month time period upon approval of the Primary Officers. The Officers' Policy shall specify the application process for Alumni membership.
\end{enumerate}

\subsection{Temporary Membership}

\subsubsection{Rights}
Temporary members of the SSE are not permitted to participate in votes, elections, or petitions. These members shall not be included in any count of the Society's members for the purposes of voting, elections, or petitions. All other rights shall be determined by the Primary Officers.

\subsubsection{Requirements}
Temporary membership may be granted for at most one academic term at the discretion of the Primary Officers if the following additional requirements are met:

\begin{enumerate}
\item The candidate must be enrolled as a student at RIT or NTID.
\item The candidate must not ever have had a Temporary, Active, or Alumni membership in the Society.
\end{enumerate}

\subsection{Removal of Membership}
All forms of membership may be contested at any time by the Primary Officers.

Membership may be removed only by a vote among the Primary Officers. Should a Primary Officer feel a member is not acting in the spirit of the SSE Constitution and its ideals, it is the responsibility of the Primary Officer to start a discussion with the other Primary Officers detailing the nature of the member's actions and reasons for removal of membership. This discussion shall be concluded with a vote among the Primary Officers for or against the placement of the candidate on a probationary period, the length of which shall be determined at the discretion of the Primary Officers based on the nature of the candidate's behavior. Should a decision in favor of probation be reached, the candidate shall be informed of the verdict and their probationary period shall begin. Should this not be reached, the candidate shall informed of the nature of the discussion.

At the end of the probationary period, a second discussion shall be held between the Primary Officers and the candidate detailing the improvement of the candidate during their probationary period. This discussion shall be concluded with a vote among the Primary Officers for or against the removal of membership from the candidate. If the officers vote in favor of removal, the candidate shall be informed of the verdict and be removed from the membership roster immediately. The former member shall also be barred from obtaining membership for a minimum of the remainder of the current academic term.

\end{document}